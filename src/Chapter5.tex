\begin{problem}{5.1}
 What is the information ratio of a passive manager?
\end{problem}

\begin{proof}[Solution]
 Passive managers will just invest in the benchmark and so their residual returns and risk will be zero. The information ratio will therefore be zero by definition.
\end{proof}

\begin{problem}{5.2}
 What is the information ratio required to add a risk-adjusted return of 2.5 precent with a moderate risk aversion level of 0.10? What level of active risk would that require?
\end{problem}

\begin{proof}[Solution]
 We want to find the IR consistent with a rigk-adjusted value added of 2.5\% and a risk aversion of 0.10. From eq. (5.12) we have
 \begin{equation*}
  \mathrm{VA^{*}}=\frac{\mathrm{IR}^{2}}{4\lambda_{R}}
 \end{equation*}
 Which implies
 \begin{align*}
  \mathrm{IR}&=2\sqrt{\lambda_{R}\mathrm{VA^{2}}}\\
	     &=2\sqrt{0.1\times2.5\%}\\
	     &=1
 \end{align*}
 From eq. (5.10) we can then calculate the active risk as
 \begin{align*}
  \omega^{*}&=\frac{\mathrm{IR}}{2\lambda_{R}}\\
	    &=\frac{1}{2*.1}\\
	    &=5\%
 \end{align*}
\end{proof}

\begin{problem}{5.3}
 Starting with the universe of MMI stocks, we make the assumptions
 \begin{itemize}
  \item[]{$Q=$ MMI portfolio}
  \item[]{$f_{q}=$ 6\%}
  \item[]{$B=$ capitalization-weighted MMI portfolio}
 \end{itemize}
 We calculate (as of January 1995) that
 \begin{center}
 \begin{tabular}{c c c c}
  \centering
  \textbf{Portfolio}	& $\bm{\beta}$ \textbf{with Respect to $B$}	& $\bm{\beta}$ \textbf{with Respect to $Q$}	& \textbf{$\bm{\sigma}$}	\\
  \hline
  $B$	& 1.000	& 0.965	& 15.50\% \\
  $Q$	& 1.004	& 1.000	& 15.82\% \\
  $C$	& 0.865	& 0.831 & 14.42\% \\
 \end{tabular}
 \end{center}
 where portfolio $C$ is the minimum-variance (fully invested) portfolio. For each portfolio ($Q$, $B$, and $C$), calculate $f$, $\alpha$, $\omega$, SR, and IR.
\end{problem}

\begin{proof}[Solution]
  For portfolio $B$ we have:
  \begin{align*}
   \alpha_{B} &= \omega_{B} = 0 \text{\quad(since B is the benchmark)} \\
   f_{B}&=\beta_{B,Q}f_{Q} + \alpha_{B}\\
        &= 0.965\times6\% -0 \\
        &= 5.79\% \\
   \mathrm{SR} &= f_{B}/\sigma_{B}\\
	       &= 5.79\%/15.50\% \\
	       &=0.374\\
   \mathrm{IR} &= \alpha_{B}/\omega_{B} \\
	       &= 0 \text{\quad(by definition)}
  \end{align*}

  
  For portfolio $Q$ we have:
  \begin{align*}
   f_{Q}&= 6\%\text{\quad(by definition)}\\
   \alpha_{Q} &= f_{Q} - \beta_{Q,B}f_{B} \\
	      &= 6\% - 1.004\times5.79\% \\
	      &= 0.187\%\\
   \omega_{Q} &= \sqrt{\sigma_{Q}^{2}-\beta_{Q,B}^{2}\sigma_{B}^{2}}\\
	      &= \sqrt{(15.82\%)^{2} -1.004^{2}\times(15.5\%)^{2}}\\
	      &= 2.85\%\\
   \mathrm{SR} &= f_{Q}/\sigma_{Q} \\
	       &= 6\%/15.82\%\\ 
	       &=0.379\\
   \mathrm{IR} &= \alpha_{Q}/\omega_{Q}\\
	       &= 0.187\%/2.85\%\\
	       &= 0.066
  \end{align*}
  
  For portfolio $C$ we have:
  \begin{align*}
   f_{C} &= \beta_{C,B}f_{B}/\beta_{Q,B} \text{\quad (see Eq 2A.38)}\\
	 &= 0.865\times 5.79\% / 1.004 \\
	 &= 4.99\%\\
   \alpha_{C} &= f_{C} - \beta_{C,B}f_{B} \\
	      &= 4.99\% - 0.865\times5.79\% \\
	      &= -0.018\%\\
   \omega_{C} &= \sqrt{\sigma_{C}^{2}-\beta_{C,B}^{2}\sigma_{B}^{2}}\\
	      &= \sqrt{(14.42\%)^{2} -0.865^{2}\times(15.50\%)^{2}}\\
	      &= 5.31\%\\
   \mathrm{SR} &= f_{C}/\sigma_{C} \\
	       &= 4.99\%/14.42\%\\ 
	       &=0.346\\
   \mathrm{IR} &= \alpha_{C}/\omega_{C}\\
	       &= -0.018\%/5.31\%\\
	       &= -0.0034\%
  \end{align*}
\end{proof}


\begin{problem}{5.4}
 You have a residual risk aversion of $\lambda_{R}=0.12$ and an information ratio of $\mathrm{IR}=0.60$. What is your optimal level of residual risk? What is your optimal value added?
\end{problem}

\begin{proof}[Solution]
 From eq. (5.10), the optimal residual risk is:
 \begin{align*}
  \omega^{*}&=\frac{\mathrm{IR}}{2\lambda_{R}}\\
	    &=\frac{0.60}{2\times 0.12}\\
	    &=2.5\%
 \end{align*}
 The optimal value added is (see eq. (5.12)):
 \begin{align*}
  \mathrm{VA^{*}}&=\frac{\mathrm{IR}^{2}}{4\lambda_{R}}\\
		 &=\frac{0.60^{2}}{4\times0.12}\\
		 &=0.75\%
 \end{align*}
\end{proof}

\begin{problem}{5.5}
 Oops. In fact, your information ratio is really only $\mathrm{IR}=0.30$. How much value added have you lost by setting your residual risk level according to Problem 4 instead of at its correct optimal level?
\end{problem}

\begin{proof}[Solution]
 The optimal residual risk for $\mathrm{IR}=0.30$ is
 \begin{align*}
  \omega^{*}&=\frac{\mathrm{IR}}{2\lambda_{R}}\\
	    &=\frac{0.30}{2\times 0.12}\\
	    &=1.25\%
 \end{align*}
 The value added using the optimal residual risk is
 \begin{align*}
  \mathrm{VA^{*}}&=\frac{\mathrm{IR}^{2}}{4\lambda_{R}}\\
		 &=\frac{0.30^{2}}{4\times0.12}\\
		 &=0.1875\%
 \end{align*}
 The value added using the residual risk from problem 4 is (see eq. (5.9)
 \begin{align*}
  \mathrm{VA}[\omega]&=\omega\cdot\mathrm{IR}-\lambda_{R}\cdot \omega^{2}\\
		     &=2.5\%\times 0.30 - 0.12\times(2.5\%)^{2}\\
		     &=0
 \end{align*}
 Hence, by using the non-optimal residual risk from problem 4, we loose 0.1875\% value added.
\end{proof}

\begin{problem}{5.6}
 You are an active manager with an information ratio of $\mathrm{IR}=0.50$ (top quartile) an a target level of residual risk of 4 percent. What residual risk aversion should lead to that level of risk?
\end{problem}

\begin{proof}[Solution]
 From eq. (5.11) we have
 \begin{align*}
  \lambda_{R}&=\frac{\mathrm{IR}}{2\omega^{*}}\\
	     &=\frac{0.50}{2\times 4\%}\\
	     &=0.0625/\%
 \end{align*}

\end{proof}



