\begin{problem}{13.1}
  Your research has identified a monthly signal with IR=1. You notice that delaying its implementation by one quarter reduces the IR to 0.75. What is the signal's half-life? What is the half-life of the value added?
\end{problem}

\begin{proof}[Solution]
  We know that $\mathrm{IR}(t=0)=1$ and $\mathrm{IR}(t=3)=0.75$. We can also write $\mathrm{IR}(t=n)=\mathrm{IR}(t=0)\delta^{n}$ where $n$ is the number of months we delay implementation. This implies that $\delta=(0.75)^{1/3}$. The half life is the number of months it takes for $\mathrm{IR}(t=n)=0.5\cdot \mathrm{IR}(t=0)$. Hence, we can write $1/2 = \delta^{\tau}$ where $\tau$ is the half life. Solving for $\tau$ we find $\tau = 3\ln(1/2)/\ln(0.75) = 7.2$ months. Furthermore, we know that the value added is proportional to the square of the IR. Hence, we can write $1/2 = \mathrm{VA}(t=0)/\mathrm{VA}(t=n) \propto (\mathrm{IR}(t=0)/\mathrm{IR}(t=n))^{2}=\delta^{2\tau}$. $\delta$ is the same as before, and so we find that the half life is $\tau=(3/2) (\ln(1/2)/\ln(3/4))$ which is half the half life of the information ratio, or about 3.6 months.  
\end{proof}

\begin{problem}{13.2}
  In further researching the signal in Problem 13.1, you discover that the correlation of active returns to this signal and this signal implemented 1 month late is 0.75. What is the optimal combination of current and lagged portfolios?
\end{problem}

\begin{proof}[Solution]
  From equations 13.1 and 13.2, the optimal weight of the current portfolio is given by
  \begin{align*}
   w_{Now}^{*} 
	       &= \frac{\delta+\frac{1-\delta}{1-\rho}}{\delta+1}\\
	       &= \frac{0.75^{1/3}+\frac{1-0.75^{1/3}}{1-0.75}}{0.75^{1/3}+1}\\
	       &= 0.67
  \end{align*}
  The weight of the lagged portfolio $w_{Later}^{*}=1-w_{Now}^{*}$ is 0.33. Because $\rho < \delta$ we can combine the lagged portfolio with the current portfolio to diversify our holdings. In effect, this reinforces the signal while damping the noise.

\end{proof}

\begin{problem}{13.3}
  You forecast $\alpha=2$ percent for a stock with $\omega=25$ percent, based on a signal with $\mathrm{IC}=0.05$. Suddenly the stock moves, with $\theta=10$ percent. How should you adjust your alpha? Is it now positive or negative?
\end{problem}

\begin{proof}[Solution]
  We need to settle the old score. Using the forecasting rule of thumb, we can determine the old score as $s(-\Delta t)=\alpha/\mathrm{IC}\cdot \omega=0.02/(0.05\times 0.25)=1.6$. Using equation 13.12, we can settle the old score according to
  \begin{equation*}
   s^{*}(-\Delta t)=s(-\Delta t) - \frac{\mathrm{IC}\cdot r(-\Delta t,0)}{\sigma\sqrt{\Delta t}}
  \end{equation*}
  Assuming $\Delta t = 1$ we have
  \begin{align*}
   s^{*}(-\Delta t)&=1.6 - \frac{0.05\times 0.1}{0.25}\\
		   &=1.58
  \end{align*}
  The revised $\alpha$ can then be calculated using the forecasting rule of thumb with the settled score as
  \begin{align*}
   \alpha^{*}(- \Delta t ) &= \omega \cdot \mathrm{IC} \cdot s^{*}(-\Delta t) \\
			   &= 0.25 \times 0.05 \times 1.58 \\
			   &= 0.01975
  \end{align*}
  So the revised $\alpha$ hardly changes from the original, which makes sense because $\theta$ is much less than a standard deviation away from the original prediction of $\alpha$. 


\end{proof}