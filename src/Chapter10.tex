\begin{problem}{10.1}
  Assume that residual returns are uncorrelated, and that we will use an optimizer to maximize risk-adjusted residual return. Using the data in Table 10.3, what asset will the optimizer choose as the largest positive active holding? How would that change if we had assigned $\alpha=1$ for buys and $\alpha=-1$ for sells? \textit{Hint:} At optimality, assuming uncorrelated residual returns, the optimal active holdings are
  \begin{align*}
   h_{n} = \frac{\alpha_{n}}{2\lambda_{R}\omega_{n}^{2}}
  \end{align*}

\end{problem}

\begin{proof}[Solution]
\end{proof}

\begin{problem}{10.2}
  For the situation described in Problem 1, show that using the forecasting rule of thumb, we assume equal risk for each asset. What happens if we just use $\alpha=1$ or buys and $\alpha=-1$ for sells?
\end{problem}

\begin{proof}[Solution]
\end{proof}


\begin{problem}{10.3}
  Use the basic forecasting formula [Eq. (10.1)] to derive Eq. (10.20), the refined forecast in the case of one asset and two forecasts.
\end{problem}

\begin{proof}[Solution]
\end{proof}


\begin{problem}{10.4}
  In the case of two forecasts [Eq. (10.20)], what is the variance of the combined forecast? What is its covariance with the return? Verify explicitly that the combination of $g$ and $g'$ in the example leads to an IC of 0.1090. Compare this to the result from Eq. (10.23).
\end{problem}

\begin{proof}[Solution]
\end{proof}


\begin{problem}{10.5}
  You are using a neural net to forecast returns to one stock. The net inputs include fundamental counting data, analyst's forecasts, and past returns. The net combines these nonlinearly. How would the forecasting rule of thumb change under these circumstances?
\end{problem}

\begin{proof}[Solution]
\end{proof}

